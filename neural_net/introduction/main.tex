\input{../preamble.tex}


\begin{document}

\section{1. Основы линейной алгебры}
\subsection{
Введение в базовую линейную алгебру
}

\textbf{Линейная алгебра}

Итак, приступим к изучению линейной алгебры. Начнём с векторов.
Нам подойдёт такое определение: вектор --- это набор из нескольких чисел, 
записанных в столбик и заключённых в круглые или квадратные скобки.
Рассмотрим пример: v=e1,e2v=e1​,e2​ и так далее enen​.
$$
v = \begin{bmatrix}
e_{1}\\
e_{2}\\
\vdots\\
e_{n}
\end{bmatrix}
$$
Размерностью вектора будет называться количество элементов в данном наборе.
Для вектора v это будет число nn.
Размерность вектора — это натуральное число.
\begin{equation}
n \in \mathbb{N}
\end{equation}
На элементы вектора таких строгих ограничений мы не накладываем, и они могут быть натуральными, целыми, рациональными, вещественными или и вовсе комплексными числами.

Векторы

\begin{enumerate}
\item Умножение на константу
\item Сложение векторов
\item Вычитание векторов
\item Взятие длины вектора
\item Скалярное произведение
\item Векторное произведение
\end{enumerate}

Итак, мы узнали что такое векторы. Давайте рассмотрим, что с ними можно делать.
Пусть у нас есть два вектора --- aa и bb. Один из них состоит из элементов 
x1…xnx1​…xn​, другой --- состоит из элементов y1…yny1​…yn​.\\
\begin{equation}
a = \begin{bmatrix}
x_{1}\\
\vdots\\
x_{n}
\end{bmatrix}
&
b = \begin{bmatrix}
y_{1}\\      
\vdots\\
y_{n}
\end{bmatrix}
\end{equation}
Ну, во-первых, любой вектор всегда можно умножить на константу.
Пусть у нас есть какая-то константа cc.
Тогда произведение вектора a и константы c будет просто равно новому вектору, 
в котором каждый элемент вектора a просто домножим на эту константу.
\begin{equation}
a \cdot c = \begin{bmatrix}
x_{1} \cdot c\\
\vdots\\
x_{n} \cdot c
\end{bmatrix}
\end{equation}
Далее.
Сложение векторов.
Если размерности двух векторов совпадают, то мы можем рассмотреть их сумму.
Сумма векторов aa и bb будет равна вектору, в котором каждый элемент равен сумме соответствующих элементов слагаемых.
То есть в нашем случае это x1+y1,x2+y2x1​+y1​,x2​+y2​ и так далее xn+ynxn​+yn​.
\begin{equation}
a + b = \begin{bmatrix}
x_{1} + y_{1}\\
\vdots\\
x_{n} + y_{n}
\end{bmatrix}
$$
Аналогично, для разности: разность векторов a−ba−b будет равна поэлементным разностям… вектору, составленному из поэлементных разностей.
Итак, это будет x1−y1x1​−y1​ и так далее xn−ynxn​−yn​.
\begin{equation}
a - b = \begin{bmatrix}
x_{1} - y_{1}\\
\vdots\\
x_{n} - y_{n}
\end{bmatrix}
\end{equation}
Далее.
Мы можем вычислить длину любого вектора.
Это сделать достаточно просто.
Длиной вектора называется число, равное корню из суммы квадратов его элементов.
Обозначается это следующим образом:
\begin{equation}
|a| = \sqrt{x_{1}^2 + x_{2}^2 + \ldots + x_{n}^2}
\end{equation}
Длина вектора a равна соответственно корню из x12+x22+x12​+x22​+ и так далее +xn2+xn2​.
Теперь рассмотрим скалярное произведение.
Скалярное произведение двух векторов равно сумме попарных произведений соответствующих элементов, то есть элементов с соответствующими индексами.
Я не упомянул об этом раньше: индексом элемента называется просто его номер.
То есть у x1x1​ индекс равен единице, например, а у xnxn​ он равен nn.
Так вот, скалярное произведение векторов aa и bb — это сумма произведений элементов с соответствующими индексами.
То есть x1⋅y1+x2⋅y2x1​⋅y1​+x2​⋅y2​ и так далее xn⋅ynxn​⋅yn​.
\begin{equation}
a \cdot b = x_{1} \cdot y_{1} + x_{2} \cdot y_{2} + \ldots + x_{n} \cdot y_{n}
\end{equation}
Что удивительно, что скалярное произведение в то же время равно произведению длин этих векторов, умноженному на косинус угла между ними.
\begin{equation}
a \cdot b = x_{1} \cdot y_{1} + x_{2} \cdot y_{2} + \ldots + x_{n} \cdot y_{n} = |a| \cdot |b| \cdot \cos \alpha
\end{equation}
Помимо скалярного произведения, результатом которого является число или скаляр существует также и векторное произведение, результатом которого является вектор.
Обозначается оно как a×ba×b.
\begin{equation}
a \times b
\end{equation}
В нашем курсе мы им не будем пользоваться, поэтому и сейчас не станем останавливаться подробнее.
Просто имейте ввиду, что такая запись обозначает векторное произведение и ни в коем случае не путайте его со скалярным.
\end{document}

